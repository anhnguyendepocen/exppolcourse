\documentclass[12pt,a4paper]{article}
\usepackage[margin=1in]{geometry}
\usepackage{setspace}
\PassOptionsToPackage{hyphens}{url}
\usepackage{hyperref}
\usepackage{mdwlist}

\usepackage{minitoc}
\dosecttoc
\setcounter{secttocdepth}{2}
\renewcommand{\stctitle}{}
\nostcpagenumbers

\setlength{\marginparwidth}{.5in}
\usepackage{natbib}
\usepackage{bibentry}
\newcommand{\reading}[2][]{\noindent -- {#1 }\bibentry{#2}.\vspace{.25em}\\}
\newcommand{\textbook}[1]{\noindent -- {#1} from Glennerster and Takavarasha.\vspace{.5em}\\} % textbook reference
\newcommand{\seealso}{\noindent \emph{See Also:}\\}
%\newcommand{\topic}[1]{\noindent \textbf{#1}\\}

\usepackage[T1]{fontenc}
\usepackage{lmodern}
\hypersetup{
    unicode=false,          % non-Latin characters in Acrobat’s bookmarks
    pdftoolbar=true,        % show Acrobat’s toolbar?
    pdfmenubar=true,        % show Acrobat’s menu?
    pdffitwindow=false,     % window fit to page when opened
    pdfstartview={FitH},    % fits the width of the page to the window
    pdftitle={Syllabus for Experimental Politics (GV319)},    % title
    pdfauthor={Thomas J. Leeper},     % author
    pdfsubject={Political Science},   % subject of the document
    pdfkeywords={politics} {rct} {randomized experiment} {GV319} {Government}, % list of keywords
    pdfnewwindow=true,      % links in new window
    pdfborder={0 0 0}
}

\begin{document}
\nobibliography*
\faketableofcontents

\begin{center}
{\Large
\noindent \textbf{GV319: Experimental Politics}
}
\end{center}
\vspace{1em}

\noindent
\textit{Instructor:}\\
Thomas J. Leeper\\
Office: CON 4.11\\
Office hours: By appointment via LfY\\
Email: \href{mailto:t.leeper@lse.ac.uk}{t.leeper@lse.ac.uk}

\vspace{1em}

\noindent \textit{Course website:}\\ \url{https://moodle.lse.ac.uk/course/view.php?id=5709} \\
\noindent \textit{Reading list:}\\ \url{http://readinglists.lse.ac.uk/lists/BA9D65E3-F764-8018-1883-4587DCB78F4F.html}\\

\noindent The purpose of this course is to develop students' ability to critically analyse and evaluate the use of randomized controlled trials (RCTs) or ``experiments'' to develop evidence-based claims about politics. The course will introduce students to the use of experiments or randomized controlled trials (RCTs) in politics to evaluate policies, programmes, and theories, including the philosophical and statistical foundations of the method, as well as ethical, normative, and practical limitations of experimentation. The course will introduce the art, science, and ethics of experimentation, debate the validity and utility of experiments as a tool of evaluation and as the basis for policymaking, and examine the findings of experimental research in five distinct political and other real-world domains, possibly including:

\begin{enumerate*}
\item Voter mobilization
\item Campaign messaging
\item Media influence
\item Social media
\item Poverty alleviation
\item Education
\item Policy nudges
\item Judgement and decision-making
\item Wages and taxation
\item Political representation
\item Political conflict
\item Legislatures
\item Public health
\item Small-group deliberation
\end{enumerate*}

\noindent The specific set of topics discussed in the course will depend on student interest drawn from the above topics (and others discussed on the first day of class).

\subsection*{Prerequisites and Availability}

Familiarity with basic algebra required and comfort with basic statistics as covered by GV249 Research Design in Political Science, or an equivalent course in research design or introductory statistics (such as ST102, ST107, ST108, GY140, SA201), is recommended.

\subsection*{Course Availability}

This course is available on the BSc in Government, BSc in Government and Economics, BSc in Government and History, BSc in Philosophy, Politics and Economics, BSc in Politics and International Relations, and BSc in Politics and Philosophy.

\section{Objectives and Evaluation}

After this course, students should be able to:

\begin{enumerate*}
\item Describe the logic of randomized experimentation for studying causal effects of interventions in comparison to other approaches.
\item Evaluate the strengths, weaknesses, and ethics of experiments as a research design and evaluation method.
\item Analyse the use and utility of experimental methods in real-world cases.
\item Apply the logic of experimental methods to political science research questions.
\end{enumerate*}

\noindent These objectives will be achieved through in-class and out-of-class solo and group activities, class discussions, and engagement with lecture and reading material. Achievement will be evaluated --- and feedback provided on those evaluations --- in the manner described next.

\subsection{Summative Assessment: Exam and Essay}

The assessment for the course comes in two parts:

\begin{enumerate}
\item An independent, 2,250-word research essay in the form of either (a) a research design proposal or (b) a case study evaluating the use of randomised experiments in an applied context.
\item A 90-minute exam during ST that will evaluate students' knowledge of course content, including statistical foundations of experimental research, how to draw inferences from randomised experiments, ethical issues, and knowledge of the various applications discussed in the course.
\end{enumerate}

\noindent Each part is weighted equally in the final grade (50\%).

The individual essay will provide students an opportunity to achieve learning outcomes (3) and (4) in greater depth, by considering either a hypothetical application in the form of a research design paper that outlines the elements of an experimental research project (namely a research question, theoretical contribution, testable hypotheses, description of the proposed data collection and analysis, ethical considerations, and policy implications) or, alternatively, a critical case study on a given application of randomised experiments in an applied setting that analyses the context and use of experiments in a real-world case.

The material covered by the exam will be drawn explicitly and directly from lectures and readings, with class sessions providing both hands-on experience with statistical aspects and discussion of substantive topics. The exam will be designed to assess learning outcomes (1--4) and a formative problem set will provide an opportunity for feedback with respect to learning outcomes (1--2).

The essay is due via Moodle on \textbf{5 December 2017}.\footnote{We will discuss this in Week 1. An alternative to which we could agree, if unanimous among students, would be 16 January 2018.} The essay should comply with LSE and Government Department policies on summative work. All summative work is subject to automatic plagiarism detection checks. Appropriate academic referencing (quotations, parenthetical citations, footnotes or endnotes, and bibliography) is required. LSE Life can provide support on academic writing and referencing.


\subsection{Formative Activities and Assessment}

Formative assessment consists of in-class discussions, a quantitative problem set (covering material from the first weeks of the course), and a presentation of students' final essay topics near the end of MT (Weeks 9 and 10). Instructor feedback will be provided on the problem set within two weeks. Peer and instructor feedback will be provided on the presentations immediately.


\subsection{Plagiarism and Academic Dishonesty}

Formative and summative coursework must comply with LSE's policies on academic misconduct and plagiarism. Among other things, ``All work for classes and seminars (which could include, for example, written assignments, group work, presentations, and any other work, including computer programs) must be the student's own work. Direct quotations from other work must be placed properly within quotation marks or indented and must be cited fully. All paraphrased material must be clearly acknowledged. Infringing this requirement, whether deliberately or not, or passing off the work of others as the student's own work, whether deliberately or not, is plagiarism.'' See the \href{http://www.lse.ac.uk/resources/calendar/undergraduate.htm}{LSE Calendar} for more information.



\section{Reading List}

Students should purchase or otherwise obtain a copy of the following required textbook:\\

\noindent Glennerster and Takavarasha. 2013. \textit{Running Randomized Evaluations: A Practical Guide}. Princeton, NJ: Princeton University Press.\\

\noindent The text is available as an ebook or for online viewing via Dawsonera:\\

\url{https://www.dawsonera.com/readonline/9781400848447}.\\

\noindent Other required readings are listed below and provided online via ReadingLists@LSE:\\

\url{http://readinglists.lse.ac.uk/lists/BA9D65E3-F764-8018-1883-4587DCB78F4F.html}

\vspace{1em}

\noindent Other books on experimental methods that students may find useful as a reference include:\\

\reading[]{Druckmanetal2011}
\reading[]{ShadishCookCampbell2001}
\reading[]{GerberGreen2012}
\reading[]{MortonWilliams2010}
\reading[]{Teele2014}
\reading[]{Desposato2015}
\reading[]{Dunning2012}
\reading[]{ImbensRubin2015}
\reading[]{MorganWinship2015}
\reading[]{AngristPischke2008}
\reading[]{JamesJilkeVanRyzin2017}

\clearpage
\section{Course Outline}

The course will meet at the following times and locations:

\begin{itemize}
\item Lecture: MT Weeks 1--5,7--11
\item Class: MT Weeks 2--5,7--11
\item Revision session: ST Week 1
\end{itemize}

\noindent The general schedule for the course is as follows. Details on the readings for each week are provided on the following pages. All readings listed under ``See Also'' are \textit{suggested} but \textit{not required}.

\secttoc


%\clearpage
\subsection{Week 1: Introduction to Experiments}

\textbook{Ch. 1--3}

\noindent -- Bornstein, David. ``The Dawn of the Evidence-Based Budget.'' \textit{The New York Times}, 30 May 2012, \url{https://opinionator.blogs.nytimes.com/2012/05/30/worthy-of-government-funding-prove-it/}\\

\noindent -- Rutter, Tamsin. ``The rise of nudge --- the unit helping politicians to fathom human behaviour.'' \textit{The Guardian}, 23 July 2015, \url{https://www.theguardian.com/public-leaders-network/2015/jul/23/rise-nudge-unit-politicians-human-behaviour}\\


% distinction b/w observational and experimental research
% history
	% evidence-based medicine
	% evidence-based policy
% types of experiments (field/survey/lab)
% administrative stuff
	% course overview
	% syllabus
	% introductions
	% topic selection

\seealso

\reading[]{Druckmanetal2006}
\reading[]{John2017}



%\clearpage
\subsection{Week 2: Statistical Foundations I}

\textbook{Ch. 4 (only pp. 141--179) and Ch. 5}

% philosophy of causal inference
% mean-difference and three common estimators: t-test, regression, and ANOVA
% standard errors
% sampling and representativeness (briefly)

% lab activity:
	% randomization
	% the randomization distribution
	% estimators: t-test, ols, and anova

\seealso

\reading[]{Holland1986}
\reading[]{Mutz2011}


%\clearpage
\subsection{Week 3: Statistical Foundations II}

\textbook{Ch. 6 and Ch. 8}
\reading[]{BanerjeeColeDufloLinden2007}

% power and MDE
% noncompliance and intention-to-treat effects
% panels and attrition
% clustering
% using covariates (briefly)
	% effect heterogeneity
	% blocking/stratification

% lab activity:
	% power calculations
	% ITT, etc.
	% attrition

%\clearpage
\subsection{Week 4: Practical Issues}

\textbook{Ch. 4 (only pp. 98--140), Ch. 7, and Ch. 9}

% randomization failure
% ethics
%% https://www.washingtonpost.com/news/monkey-cage/wp/2014/11/02/how-to-make-field-experiments-more-ethical/

\noindent -- Snape, Joel. ``The Holborn escalator experiment proves that we value efficiency more than our own health.'' \textit{The Telegraph}, 18 April 2016, \url{http://www.telegraph.co.uk/men/thinking-man/the-holborn-escalator-experiment-proves-that-we-value-efficiency/}.\\

\noindent -- Goel, Vindu. ``Facebook Tinkers With Users' Emotions in News Feed Experiment, Stirring Outcry.'' \textit{The New York Times}, 29 June 2014, \url{https://www.nytimes.com/2014/06/30/technology/facebook-tinkers-with-users-emotions-in-news-feed-experiment-stirring-outcry.html}.\\


\seealso

\reading[]{Teele2014}
\reading[]{Cartwright2007}
\reading[]{CartwrightHardie2012}
\reading[]{Desposato2015}
\reading[]{ShadishCookCampbell2001}

\subsection{Week 5: The Politics of Evidence}

\textit{Problem Set Due}

\reading[]{Haynesetal2012}

-- Gage, Suzi. ``Let's help MPs understand the value of randomised controlled trials.'' \textit{The Guardian}, 13 April 2015, \url{https://www.theguardian.com/science/sifting-the-evidence/2015/apr/13/lets-help-mps-understand-the-value-of-randomised-controlled-trials}

-- Callen, Michael, Adnana Khan, Asim I. Khwaja, Asad Liaqat, and Emily Myers. ``These 3 barriers make it hard for policymakers to use the evidence that development researchers produce.'' \textit{The Washington Post}, 13 August 2017,  \url{https://www.washingtonpost.com/amphtml/news/monkey-cage/wp/2017/08/13/these-3-barriers-make-it-hard-for-policymakers-to-use-the-evidence-that-development-researchers-produce/}





%\clearpage
\subsection{Week 6: Reading Week}

No lecture or class.

%\clearpage
\subsection{Week 7: Substantive Topic 1}

\noindent For Weeks 7--11, topics of discussion will be determined based upon in-class discussion in Weeks 1--2.

\vspace{1em}

\noindent Feedback on Problem Set returned.

%\clearpage
\subsection{Week 8: Substantive Topic 2}

\vspace{1em}

\begin{itemize*}
\item 1-minute ``elevator pitch'' of proposal topics.
\item Students should sign-up for presentation slots.
\end{itemize*}

%\clearpage
\subsection{Week 9: Substantive Topic 3}

\textit{Student presentations in-class this week.}

%\clearpage
\subsection{Week 10: Substantive Topic 4}

\textit{Student presentations in-class this week.}

%\clearpage
\subsection{Week 11: Substantive Topic 5 and Conclusion}

\reading[]{DeatonCartwright2016}

\seealso

\reading[]{Ogden2017}
\reading[]{Cartwright2007}
\reading[]{CartwrightHardie2012}


\subsection{ST Revision Session}

One-hour session to discuss final questions about ST exam.

\clearpage
\section{Relevant Resources}

\small

\subsection*{Discussions of experiments in the popular press}

\begin{itemize}
\item Callen, Michael, Adnana Khan, Asim I. Khwaja, Asad Liaqat, and Emily Myers. ``These 3 barriers make it hard for policymakers to use the evidence that development researchers produce.'' \textit{The Washington Post}, 13 August 2017,  \url{https://www.washingtonpost.com/amphtml/news/monkey-cage/wp/2017/08/13/these-3-barriers-make-it-hard-for-policymakers-to-use-the-evidence-that-development-researchers-produce/}

\item Jauhiainen, Antti and M\"{a}kinen, Joona-Hermanni. ``Why Finland's Basic Income Experiment Isn't Working.'' \textit{The New York Times}, 20 July 2017, \url{https://www.nytimes.com/2017/07/20/opinion/finland-universal-basic-income.html}

\item ``Policymakers around the world are embracing behavioural science.'' \textit{The Economist}, 18 May 2017, \url{https://www.economist.com/news/international/21722163-experimental-iterative-data-driven-approach-gaining-ground-policymakers-around}

\item Soumeri, Stephen B., and Koppel, Ross. ``Paying doctors bonuses for better health outcomes makes sense in theory. But it doesn't work.'' \textit{Vox.com}, 25 January 2017, \url{https://www.vox.com/the-big-idea/2017/1/25/14375776/pay-for-performance-doctors-bonuses}

\item Free Exchange. ``Economists are prone to fads, and the latest is machine learning.'' \textit{The Economist}, 24 November 2016, \url{https://www.economist.com/news/finance-and-economics/21710800-big-data-have-led-latest-craze-economic-research-economists-are-prone}

\item Mullainathan, Sendhil. ``Ban the Box? An Effort to Stop Discrimination May Actually Increase It.'' \textit{The New York Times}, 19 August 2016, \url{https://www.nytimes.com/2016/08/21/upshot/ban-the-box-an-effort-to-stop-discrimination-may-actually-increase-it.html}

\item ``Can mass media cause change? A randomised control trial finds out.'' \textit{BBC Media Action Insight Blog}, 14 July 2016, \url{http://www.bbc.co.uk/blogs/mediaactioninsight/entries/703ec6e0-e8e1-4891-b614-752d48c678fc}

\item Kushner, Jacob. ``Can science save development aid?'' \textit{Pacific Standard}, 6 July 2016, \url{ https://psmag.com/news/can-science-save-development-aid}

\item ``The Holborn escalator experiment proves that we value efficiency more than our own health.'' \textit{The Telegraph}, 18 April 2016, \url{http://www.telegraph.co.uk/men/thinking-man/the-holborn-escalator-experiment-proves-that-we-value-efficiency/}.

\item Porter, Eduardo. ``Nudge Aren't Enough for Problems like Retirement Savings.'' \textit{The New York Times}, 23 February 2016, \url{https://www.nytimes.com/2016/02/24/business/economy/nudges-arent-enough-to-solve-societys-problems.html}

\item Matthews, Dylan. ``Economists tested 7 welfare programs to see if they made people lazy. They didn't.'' \textit{Vox.com}, 20 November 2015, \url{https://www.vox.com/policy-and-politics/2015/11/20/9764324/welfare-cash-transfer-work}

\item Whoriskey, Peter. ``The science of skipping breakfast: How government nutritionists may have gotten it wrong.'' \textit{The Washington Post}, 10 August 2015, \url{https://www.washingtonpost.com/news/wonk/wp/2015/08/10/the-science-of-skipping-breakfast-how-government-nutritionists-may-have-gotten-it-wrong/}

\item ``The rise of nudge --- the unit helping politicians to fathom human behaviour.'' \textit{The Guardian}, 23 July 2015, \url{https://www.theguardian.com/public-leaders-network/2015/jul/23/rise-nudge-unit-politicians-human-behaviour}.

\item Meyer, Michelle N. and Chabris, Christopher F. ``Please, Corporations, Experiment on Us.'' \textit{The New York Times}, 19 June 2015, \url{https://www.nytimes.com/2015/06/21/opinion/sunday/please-corporations-experiment-on-us.html}

\item Johnson, Jeremy. ``Campaign experiment found to be in violation of Montana law.'' \textit{The Washington Post}, 13 May 2015, \url{https://www.washingtonpost.com/news/monkey-cage/wp/2015/05/13/campaign-experiment-found-to-be-in-violation-of-montana-law/}

\item Gage, Suzi. ``Let's help MPs understand the value of randomised controlled trials.'' \textit{The Guardian}, 13 April 2015, \url{https://www.theguardian.com/science/sifting-the-evidence/2015/apr/13/lets-help-mps-understand-the-value-of-randomised-controlled-trials}

\item Frakt, Austin. ``Alcoholics Anonymous and the Challenge of Evidence-Based Medicine.'' \textit{The New York Times}, 6 April 2015, \url{https://www.nytimes.com/2015/04/07/upshot/alcoholics-anonymous-and-the-challenge-of-evidence-based-medicine.html}

\item Mullainathan, Sendhil. ``Racial Bias, Even When We Have Good Intentions.'' \textit{The New York Times}, 3 January 2015, \url{https://www.nytimes.com/2015/01/04/upshot/the-measuring-sticks-of-racial-bias-.html}

\item Bernard, Tara Siegel. ``A Citizen's Guide to Buying Political Access.'' \textit{The New York Times}, 18 November 2014, \url{https://www.nytimes.com/2014/11/19/your-money/a-citizens-guide-to-buying-political-access-.html}

\item Willis, Derek. ``Professors' Research Project Stirs Political Outrage in Montana.'' \textit{The New York Times}, 28 October 2014, \url{https://www.nytimes.com/2014/10/29/upshot/professors-research-project-stirs-political-outrage-in-montana.html}

\item Boseley, Sarah. ``Ebola vaccine trials with placebo group would be unethical, scientists say.'' \textit{The Guardian}, 10 October 2014, \url{https://www.theguardian.com/world/2014/oct/10/ebola-vaccine-placebo-trials-unethical-scientists-say}

\item Ensor, Josie. ``Dating site OKCupid admits to Facebook-style psychological testing on users.'' \textit{The Telegraph}, 29 July 2014,  \url{http://www.telegraph.co.uk/news/worldnews/northamerica/usa/10996866/Dating-site-OKCupid-admits-to-Facebook-style-psychological-testing-on-users.html}

\item Matthews, Dylan. ``A guaranteed income for every American would eliminate poverty — and it wouldn't destroy the economy.'' \textit{Vox.com}, 23 July 2014, \url{https://www.vox.com/2014/7/23/5925041/guaranteed-income-basic-poverty-gobry-labor-supply}

\item Rosen, Jay. ``Facebook's controversial study is business as usual for tech companies but corrosive for universities.'' \textit{Washington Post}, 3 July 2014, \url{https://www.washingtonpost.com/posteverything/wp/2014/07/03/dont-blame-facebook-for-screwing-with-your-mood-blame-academia/}.

\item ``Facebook Tinkers With Users' Emotions in News Feed Experiment, Stirring Outcry.'' \textit{The New York Times}, 29 June 2014, \url{https://www.nytimes.com/2014/06/30/technology/facebook-tinkers-with-users-emotions-in-news-feed-experiment-stirring-outcry.html}.

\item Konnikova, Maria. ``I don't want to be right.'' \textit{The New Yorker}, 16 May 2014, \url{http://www.newyorker.com/science/maria-konnikova/i-dont-want-to-be-right/amp}
\item Issenberg, Sasha. ``Dept. of Experiments.'' \textit{Politico}, 27 February 2014, \url{http://www.politico.com/magazine/story/2014/02/campaign-science-dept-of-experiments-103671_Page4.html#.WYxvh1WGM4l}

\item Free Exchange. ``Random harvest.'' \textit{The Economist}, 14 December 2013, \url{https://www.economist.com/news/finance-and-economics/21591573-once-treated-scorn-randomised-control-trials-are-coming-age-random-harvest}

\item Bennhold, Katrin. ``Britain' Ministry of Nudges.'' \textit{The New York Times}, 7 December 2013, \url{http://www.nytimes.com/2013/12/08/business/international/britains-ministry-of-nudges.html}

\item Swinford, Steven. ``Environment minister: 'The badgers have moved the goalposts'.'' \textit{The Telegraph}, 9 October 2013, \url{http://www.telegraph.co.uk/news/earth/environment/10367057/Environment-minister-The-badgers-have-moved-the-goalposts.html}

\item Matthews, Dylan. ``Teach for America is a deeply divisive program. It also works.'' \textit{The Washington Post}, 10 September 2013, \url{https://www.washingtonpost.com/news/wonk/wp/2013/09/10/teach-for-america-is-a-deeply-divisive-program-it-also-works/}

\item Smith, Marc. ``Evidence-based education: is it really that straightforward?'' \textit{The Guardian}, 2 March 2013, \url{https://www.theguardian.com/teacher-network/2013/mar/26/teachers-research-evidence-based-education}

\item Issenberg, Sasha. ``The Death of a Hunch.'' \textit{Slate.com}, 22 May 2012,  \url{http://www.slate.com/articles/news_and_politics/victory_lab/2012/05/obama_campaign_ads_how_the_analyst_institute_is_helping_him_hone_his_message_.html}

\item Henderson, Mark. ``Using the tools of science to improve social policy.'' \textit{The Guardian}, 13 May 2012, \url{https://www.theguardian.com/society/2012/may/13/scientific-method-test-public-policy}

\item Bornstein, David. ``The Dawn of the Evidence-Based Budget.'' \textit{The New York Times}, 30 May 2012, \url{https://opinionator.blogs.nytimes.com/2012/05/30/worthy-of-government-funding-prove-it/}

\item Christian, Brian. ``The A/B Test: Inside the Technology That's Changing the Rules of Business.'' \textit{Wired}, 25 April 2012, \url{https://www.wired.com/2012/04/ff_abtesting/}

\item Free Exchange. ``The gain from early intervention.'' \textit{The Economist}, 24 October 2011, \url{https://www.economist.com/blogs/freeexchange/2011/10/education}

\item Goldacre, Ben. ``If you want answers, why not run your own trials?'' \textit{The Guardian}, 30 September 2011, \url{https://www.theguardian.com/commentisfree/2011/sep/30/run-your-own-scientific-trials}

\item Issenberg, Sasha. ``Nudge the Vote.'' \textit{The New York Times Magazine}, 29 October 2010, \url{http://www.nytimes.com/2010/10/31/magazine/31politics-t.html}

\item Parker, Ian. ``The Poverty Lab.'' \textit{The New Yorker}, 10 May 2010, \url{http://www.newyorker.com/magazine/2010/05/17/the-poverty-lab/amp}

\item Leonhardt, David. ``Sometimes, What's Needed Is a Nudge.'' \textit{The New York Times}, 16 May 2007, \url{http://www.nytimes.com/2007/05/16/business/16leonhardt.html}

\end{itemize}


\subsection*{Organisations using RCTs and providing resources for experimental evaluations}

\begin{itemize}

\item What Works Network (UK Cabinet Office): \url{https://www.gov.uk/guidance/what-works-network}

\item Evidence in Governance and Politics: \url{http://egap.org/}

\item Jameel Poverty Action Lab: \url{https://www.povertyactionlab.org/}

\item The World Bank: \url{http://blogs.worldbank.org/impactevaluations/}

\item 3ie (International Initiative for Impact Evaluation): \url{http://www.3ieimpact.org/}

\item Innovations for Poverty Action: \url{http://www.poverty-action.org/}

\item University of California Center for Effective Global Action: \url{http://cega.berkeley.edu/}

\item The Behavioural Insights Team: \url{http://www.behaviouralinsights.co.uk/}

\item (Former) US Government Social and Behavioral Sciences Team: \url{https://sbst.gov/}

\item Mathematica Policy Research: \url{https://www.mathematica-mpr.com/}

\item Laura and John Arnold Foundation: \url{http://www.arnoldfoundation.org/initiative/evidence-based-policy-innovation/}

\item MDRC: \url{http://www.mdrc.org/}

\item Abt Associates: \url{http://abtassociates.com/}

\end{itemize}

\subsection*{Clearinghouses of Experimental Evidence}

\begin{itemize}
\item The Cochrane Collaboration: \url{http://www.cochrane.org/}

\item The Campbell Collaboration: \url{https://www.campbellcollaboration.org/}

\item US Department of Education What Works Clearinghouse: \url{https://ies.ed.gov/ncee/wwc/}

\end{itemize}


\subsection*{Examples of high-profile experiments}

\begin{itemize}

\item RAND Corporation Health Insurance Experiment: \url{https://www.rand.org/health/projects/hie.html}

\item Employment Retention and Advancement demonstration: \url{https://www.gov.uk/government/publications/employment-retention-and-advancement-demonstration-rr727}

\item ``Project STAR'', Tennessee Class Size Experiment: \url{http://www.jstor.org/stable/1163474}

\item High/Scope Perry Preschool Project: \url{https://highscope.org/perrypreschoolstudy}

\item Prospera/Oportunidades (Mexico conditional cash transfer programme): \url{https://en.wikipedia.org/wiki/Oportunidades}

\item Rothamsted agronomic experiments: \url{http://www.era.rothamsted.ac.uk/}

\item ``Milgram Experiments'' on obedience: \url{https://en.wikipedia.org/wiki/Milgram_experiment}

\item Physicians' Health Study: \url{http://phs.bwh.harvard.edu/}

\end{itemize}


\bibliographystyle{plain}
\nobibliography{References}

\end{document}
