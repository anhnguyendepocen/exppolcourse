\documentclass[12pt,a4paper]{article}
\usepackage[margin=1in]{geometry}
\usepackage{setspace}
\usepackage{hyperref}
\usepackage{mdwlist}

\usepackage{minitoc}
\dosecttoc
\setcounter{secttocdepth}{2}
\renewcommand{\stctitle}{}
\nostcpagenumbers

\setlength{\marginparwidth}{.5in}
\usepackage{natbib}
\usepackage{bibentry}
\newcommand{\reading}[2][]{\noindent -- {#1 }\bibentry{#2}.\vspace{.25em}\\}
\newcommand{\textbook}[1]{\noindent -- {#1} from Glennerster and Takavarasha.\vspace{.5em}\\} % textbook reference
\newcommand{\seealso}{\noindent \emph{See Also:}\\}
%\newcommand{\topic}[1]{\noindent \textbf{#1}\\}

\usepackage[T1]{fontenc}
\usepackage{lmodern}
\hypersetup{
    bookmarks=true,         % show bookmarks bar?
    unicode=false,          % non-Latin characters in Acrobat’s bookmarks
    pdftoolbar=true,        % show Acrobat’s toolbar?
    pdfmenubar=true,        % show Acrobat’s menu?
    pdffitwindow=false,     % window fit to page when opened
    pdfstartview={FitH},    % fits the width of the page to the window
    pdftitle={Syllabus for Experimental Politics (GV319)},    % title
    pdfauthor={Thomas J. Leeper},     % author
    pdfsubject={Political Science},   % subject of the document
    pdfkeywords={politics} {rct} {randomized experiment} {GV319} {Government}, % list of keywords
    pdfnewwindow=true,      % links in new window
    pdfborder={0 0 0}
}

\begin{document}
\nobibliography*
\faketableofcontents

\begin{center}
{\Large
\noindent \textbf{GV319: Experimental Politics}
}
\end{center}
\vspace{1em}

\noindent
\textit{Instructor:}\\
Thomas J. Leeper\\
Office: CON 4.11\\
Office hours: By appointment via LfY\\
Email: \href{mailto:t.leeper@lse.ac.uk}{t.leeper@lse.ac.uk}

\vspace{1em}

\noindent \textit{Course website:} \url{https://moodle.lse.ac.uk/course/view.php?id=5709} \\
\noindent \textit{Reading list:} \url{http://readinglists.lse.ac.uk/lists/BA9D65E3-F764-8018-1883-4587DCB78F4F.html}\\

\noindent The purpose of this course is to develop students' ability to critically analyse and evaluate the use of randomized controlled trials (RCTs) or ``experiments'' to develop evidence-based claims about politics.The course will introduce students to the use of experiments or randomized controlled trials (RCTs) in politics to evaluate policies, programmes, and theories, including the philosophical and statistical foundations of the method, as well as ethical, normative, and practical limitations of experimentation. The course will introduce the art, science, and ethics of experimentation, debate the validity and utility of experiments as a tool of evaluation and as the basis for policymaking, and examine the findings of experimental research in five distinct political and other real-world domains, possibly including:

\begin{enumerate*}
\item Voter mobilization
\item Campaign messaging
\item Media influence
\item Social media
\item Poverty alleviation
\item Education
\item Policy nudges
\item Judgement and decision-making
\item Wages and taxation
\item Political representation
\item Political conflict
\item Legislatures
\item Public health
\item Small-group deliberation
\end{enumerate*}

\noindent The specific set of topics discussed in the course will depend on student interest drawn from the following topics (and others discussed on the first day of class).

\subsection*{Prerequisites and Availability}

Familiarity with basic algebra required and comfort with basic statistics as covered by GV249 Research Design in Political Science, or an equivalent course in research design or introductory statistics (such as ST102, ST107, ST108, GY140, SA201), is recommended.

\subsection*{Course Availability}

This course is available on the BSc in Government, BSc in Government and Economics, BSc in Government and History, BSc in Philosophy, Politics and Economics, BSc in Politics and International Relations, and BSc in Politics and Philosophy.

\section{Objectives and Evaluation}

After this course, students should be able to:

\begin{enumerate*}
\item Describe the logic of randomized experimentation for studying causal effects of interventions in comparison to other approaches.
\item Evaluate the strengths, weaknesses, and ethics of experiments as a research design and evaluation method.
\item Analyse the use and utility of experimental methods in real world cases.
\item Apply the logic of experimental methods to political science research questions.
\end{enumerate*}

\noindent These objectives will be achieved through in-class and out-of-class solo and group activities, class discussions, and engagement with lecture and reading material. Achievement will be evaluated --- and feedback provided on those evaluations --- in the manner described next.

\subsection{Summative Assessment: Exam and Essay}

The assessment for the course comes in two parts:

\begin{enumerate}
\item An independent, 2,250-word research essay in the form of either (a) a research design proposal or (b) a case study evaluating the use of randomised experiments in an applied context.
\item A 90-minute exam during ST that will evaluate students' knowledge of course content, including statistical foundations of experimental research, how to draw inferences from randomised experiments, ethical issues, and knowledge of the various applications discussed in the course.
\end{enumerate}

\noindent The individual essay will provide students an opportunity to achieve learning outcomes (3) and (4) in greater depth, by considering either a hypothetical application in the form of a research design paper that outlines the elements of an experimental research project (namely a research question, theoretical contribution, testable hypotheses, description of the proposed data collection and analysis, ethical considerations, and policy implications) or, alternatively, a critical case study on a given application of randomised experiments in an applied setting that analyses the context and use of experiments in a real-world case.

The material covered by the exam will be drawn explicitly and directly from lectures and readings, with class sessions providing both hands-on experience with statistical aspects and discussion of substantive topics. The exam will be designed to assess learning outcomes (1--4) and a formative problem set will provide an opportunity for feedback with respect to learning outcomes (1--2).

The essay is due via Moodle on \textbf{XXXX}. The essay should comply with LSE and Government Department policies on summative work. All summative work is subject to automatic plagiarism detection checks. Appropriate academic referencing (quotations, parenthetical citations, footnotes or endnotes, and bibliography) is required. LSE Life can provide support on academic writing and referencing.


\subsection{Formative Activities and Assessment}

Formative assessment consists of in-class discussions, a quantitative problem set (covering material from the first weeks of the course), and a presentation of students' final essay topics near the end of MT (Weeks 9 and 10). Instructor feedback will be provided on the problem set within two weeks. Peer and instructor feedback will be provided on the presentations immediately.


\subsection{Plagiarism and Academic Dishonesty}

Formative and summative coursework must comply with LSE's policies on academic miconduct and plagiarism. Among other things, ``All work for classes and seminars (which could include, for example, written assignments, group work, presentations, and any other work, including computer programs) must be the student's own work. Direct quotations from other work must be placed properly within quotation marks or indented and must be cited fully. All paraphrased material must be clearly acknowledged. Infringing this requirement, whether deliberately or not, or passing off the work of others as the student’s own work, whether deliberately or not, is plagiarism.'' See the \href{http://www.lse.ac.uk/resources/calendar/undergraduate.htm}{LSE Calendar} for more information.



\section{Reading List}

Students should purchase or otherwise obtain a copy of the following required textbook:\\

\noindent Glennerster and Takavarasha. 2013. \textit{Running Randomized Evaluations: A Practical Guide}. Princeton, NJ: Princeton University Press.\\

\noindent The text is available as an ebook or for online viewing via Dawsonera:\\

\url{https://www.dawsonera.com/readonline/9781400848447}.\\

\noindent Other required readings are listed below and provided online via ReadingLists@LSE:\\

\url{http://readinglists.lse.ac.uk/lists/BA9D65E3-F764-8018-1883-4587DCB78F4F.html}

\vspace{1em}

\noindent Other books on experimental methods that students may find useful as a reference include:\\

\reading[]{Druckmanetal2011}
\reading[]{ShadishCookCampbell2001}
\reading[]{GerberGreen2012}
\reading[]{MortonWilliams2010}
\reading[]{Teele2014}
\reading[]{Desposato2015}
\reading[]{Dunning2012}
\reading[]{ImbensRubin2015}
\reading[]{MorganWinship2015}
\reading[]{AngristPischke2008}

\clearpage
\section{Course Outline}

Class will meet at the following times and locations:

\begin{itemize}
\item Lecture: MT Weeks 1--5,7--11
\item Class: MT Weeks 2--5,7--11
\item Revision session: ST Week 1
\end{itemize}

\noindent The general schedule for the course is as follows. Details on the readings for each week are provided on the following pages.

\secttoc


\clearpage
\subsection{Week 1: Introduction to Experiments}

\noindent -- ``The Holborn escalator experiment proves that we value efficiency more than our own health.'' \textit{The Telegraph}, 18 April 2016, \url{http://www.telegraph.co.uk/men/thinking-man/the-holborn-escalator-experiment-proves-that-we-value-efficiency/}.\\

\noindent -- ``Facebook Tinkers With Users' Emotions in News Feed Experiment, Stirring Outcry.'' \textit{The New York Times}, 29 June 2014, \url{https://www.nytimes.com/2014/06/30/technology/facebook-tinkers-with-users-emotions-in-news-feed-experiment-stirring-outcry.html}.\\

\noindent -- ``The rise of nudge --- the unit helping politicians to fathom human behaviour.'' \textit{The Guardian}, 23 July 2015, \url{https://www.theguardian.com/public-leaders-network/2015/jul/23/rise-nudge-unit-politicians-human-behaviour}.\\

\textbook{Ch. 1--3}

% distinction b/w observational and experimental research
% history
	% evidence-based medicine
	% evidence-based policy
% types of experiments (field/survey/lab)
% administrative stuff
	% course overview
	% syllabus
	% introductions
	% topic selection

\seealso

\reading[]{Druckmanetal2006}
\reading[]{John2017}



\clearpage
\subsection{Week 2: Statistical Foundations I}

\textbook{Ch. 4 (only pp. 141--179) and Ch. 5}

% philosophy of causal inference
% mean-difference and three common estimators: t-test, regression, and ANOVA
% standard errors
% sampling and representativeness (briefly)

% lab activity:
	% randomization
	% the randomization distribution
	% estimators: t-test, ols, and anova

\seealso

\reading[]{Holland1986}
\reading[]{Mutz2011}


\clearpage
\subsection{Week 3: Statistical Foundations II}

\textbook{Ch. 6 and Ch. 8}

% power and MDE
% noncompliance and intention-to-treat effects
% panels and attrition
% clustering
% using covariates (briefly)
	% effect heterogeneity
	% blocking/stratification

% lab activity:
	% power calculations
	% ITT, etc.
	% attrition

\clearpage
\subsection{Week 4: Practical Issues}

\textbook{Ch. 4 (only pp. 98--140), Ch. 7, and Ch. 9}

% randomization failure
% ethics

\seealso

\reading[]{Teele2014}
\reading[]{Cartwright2007}
\reading[]{CartwrightHardie2012}
\reading[]{Desposato2015}



\clearpage


\noindent For Weeks 5, and 7--10, topics of discussion will be determined based upon in-class discussion in Weeks 1--2.

\subsection{Week 5: Substantive Topic 1}

Problem Set Due



%\clearpage
\subsection{Week 6: Reading Week}

No lecture or class.

%\clearpage
\subsection{Week 7: Substantive Topic 2}


\vspace{1em}

\noindent Feedback on Problem Set returned.

%\clearpage
\subsection{Week 8: Substantive Topic 3}

\vspace{1em}

\begin{itemize*}
\item 1-minute ``elevator'' pitch of proposal topics.
\item Student should sign-up for presentation slots.
\end{itemize*}



%\clearpage
\subsection{Week 9: Substantive Topic 4}

\textit{Student presentations in-class this week.}

%\clearpage
\subsection{Week 10: Substantive Topic 5}

\textit{Student presentations in-class this week.}

\clearpage
\subsection{Week 11: Conclusion}


\reading[]{DeatonCartwright2016}

% strengths and weaknesses of experiments
	% Cartwright?
% questions

\seealso

\reading[]{Ogden2017}
\reading[]{Cartwright2007}
\reading[]{CartwrightHardie2012}


\subsection{ST Revision Session}

One-hour session to discuss final questions about ST exam.



\bibliographystyle{plain}
\nobibliography{References}

\end{document}
